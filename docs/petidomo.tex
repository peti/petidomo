\documentclass[a4paper,10pt]{scrreprt}
%
% Petidomo Manual
%
% $Header$
%
\typearea[2cm]{12}
\usepackage{lastpage}
\usepackage{fancyhdr}
\pagestyle{fancy}
\lhead{\sl The Petidomo Mailing List Manager}
\chead{}
\rhead{Page \thepage\ of \pageref{LastPage}}
\lfoot{}
\cfoot{}
\rfoot{}
\fancypagestyle{plain}{}
\fussy

\newcommand{\Def}[1]{{\sl #1}}
\newcommand{\file}[1]{{\sf #1}}

\begin{document}

\title{The Petidomo Mailing List Manager}
\author{Peter Simons $<$simons@computer.org$>$}
\date{February 2001}
\maketitle
\tableofcontents
\clearpage

\chapter{Installing Petidomo}

The installation of the Petidomo Mailing List Manager is simple and
straight forward; do not be scared by the length of this chapter.
There are many different ways and options how to install it and I have
tried my best to cover \emph{all} of them. If you are not interested
in every little detail, you will be able to skim over most of the text
here.

\section{Getting it}

    {\sf This section has not been written yet, because the means of
    obtaining Petidomo from the Internet are not entirely defined at
    the moment. I put all my hope into the most kind Mister
    Engelschall to fill in the details here. }

\section{Building the Binaries}

Untar the source archive of Petidomo in a directory of your choice
like {\sf /usr/local/src} or your home directory. This will create a
directory called {\sf petidomo-VERSION}, where the ``{\sf VERSION}''
part is called exactly as in the file name of the tar archive. Change
into this directory.

Now you have to run the configure script
\begin{quote}
\begin{verbatim}
./configure
\end{verbatim}
\end{quote}
which will determine the characteristics of your system and create the
files required to actually build Petidomo. You may provide several
parameters to the script. The interesting ones, including the default
values if unspecified, are:
\begin{description}

\item[{-}{-}help] Display the complete list of command line options.

\item[{-}{-}prefix] The the {\sf PREFIX} for all following paths. The
default is {\sf /usr/local}.

\item[{-}{-}exec-prefix] Set the {\sf EPREFIX} for all following
paths. This is useful in case you want to install binaries into a
different directory hierarchy than normal text files, but usually the
{\sf EPREFIX} is identical to {\sf PREFIX}. The default is {\sf
PREFIX}.

\item[{-}{-}bindir] Set the directory where the binaries should be
installed. The default is {\sf EPREFIX/bin}.

\item[{-}{-}libexecdir] Set the directory where executables should be
installed that will be called by Petidomo but not by the user directly
(like posting filters). The default is {\sf EPREFIX/libexec}.

\item[{-}{-}datadir] Set the directory where read-only
architecture-independent data files should be installed (like the help
file). The default is {\sf PREFIX/share}.

\item[{-}{-}sysconfdir] Set the directory where read-only
configuration files should be installed. The default is {\sf
PREFIX/etc}.

\item[{-}{-}localstatedir] Set the directory where modifiable
data files should be installed (like the approve-queue or the mailing
list config files). The default is {\sf PREFIX/var}.

\item[{-}{-}mandir] Set the directory where man documentation files
should be installed. The default is {\sf PREFIX/man}.

\end{description}

Please note that the directories you specify here are only the default
settings that are compiled into Petidomo. You can modify \emph{all}
paths at run-time via the command line and through the configuration
files. So don't waste to much time figuring out what you want here,
you can change anything later without having to recompile Petidomo.

Finally, here is an example output of the configuration script when
run without any parameters on a Linux machine:
\begin{quote}
\begin{verbatim}
simons@peti:~/projects/petidomo-4.0b1$ ./configure
Configuring OSSP Petidomo, Version 4.0b1 (18-Jan-2001)
creating cache ./config.cache
checking for gcc... gcc
checking whether the C compiler (gcc  ) works... yes
checking whether the C compiler (gcc  ) is a cross-compiler... no
checking whether we are using GNU C... yes
checking whether gcc accepts -g... yes
checking for ranlib... ranlib
checking for flex... flex
checking for yywrap in -lfl... yes
checking for bison... bison -y
checking size of unsigned short... 2
checking size of unsigned int... 4
checking size of unsigned long... 4
checking how to run the C preprocessor... gcc -E
checking for ANSI C header files... yes
checking for ssize_t... yes
updating cache ./config.cache
creating ./config.status
creating Makefile
\end{verbatim}
\end{quote}

Often, you may want to pass certain flags to the compiler or the
linker to modify the building process. To achieve this, you can set
certain environment variables before calling the configure script.
These variables are:
\begin{description}
\item[CC] The name of the C compiler to use.

\item[CPPFLAGS] Flags to pass to the preprocesser before compiling a
source code file.

\item[CFLAGS] Flags to pass to the compiler when compiling a C source
code file.

\item[LDFLAGS] Flags to pass to the linker when linking the binaries.

\end{description}

I personally find this useful to raise the level of compiler
optimization or to add linker flags that tell the linker to strip
unnecessary symbols from the binaries. To achive these effects, I call
the configure script like this:
\begin{quote}
\begin{verbatim}
CFLAGS=-O3 LDFLAGS=-s ./configure
\end{verbatim}
\end{quote}

Anyway, once the configure script has been run, just call
\begin{quote}
\begin{verbatim}
make
\end{verbatim}
\end{quote}
to start the building process. Petidomo has been tested with various
flavours of the make utility and all of them seem to work fine. If in
doubt, try GNU Make, which is available from {\sf ftp.gnu.org}.

Petidomo has also been built using parallel builds. This is useful if
you have a multi-processer system. You can do this with most make
utilities by adding the flag ``-j4'' with ``4'' being the number of
processes you want to spawn simultaneously. Please note, though, that
some make utilities have problems with the rules that translate the
yacc-modules included in Petidomo correctly when building in parallel.
If you experience any trouble, just build it conventionally and you
should be fine.

\section{Installing the Binaries}

To install the software to your system, all you have to do is execute
\begin{quote}
\begin{verbatim}
make install
\end{verbatim}
\end{quote}
This will copy the Petidomo binary, the posting filters included in
the distribution, the sample config files and the manual pages into
the directories you chose at configure time earlier. If you're a
first-time user, you may also want to execute
\begin{quote}
\begin{verbatim}
make install-testlist
\end{verbatim}
\end{quote}
which will create a sample mailing list called ``testlist'' for you.

Assuming you used the default paths when running configure, the
install process will create the follwing directories, respectively
copy the following files to your system:
\begin{quote}
\begin{verbatim}
/usr/local/
/usr/local/bin/
/usr/local/bin/petidomo
/usr/local/bin/petidomo-approve
/usr/local/bin/petidomo-kickout
/usr/local/etc/
/usr/local/etc/petidomo.acl-sample
/usr/local/etc/petidomo.conf-sample
/usr/local/libexec/
/usr/local/libexec/petidomo/
/usr/local/libexec/petidomo/insert-name-in-subject.sh
/usr/local/libexec/petidomo/pgp-decrypt.sh
/usr/local/libexec/petidomo/pgp-encrypt.sh
/usr/local/libexec/petidomo/rfc2369.sh
/usr/local/man/
/usr/local/man/man1/
/usr/local/man/man1/petidomo.1
/usr/local/share/
/usr/local/share/petidomo/
/usr/local/share/petidomo/help
/usr/local/var/
/usr/local/var/petidomo/
/usr/local/var/petidomo/ack-queue/
/usr/local/var/petidomo/index
/usr/local/var/petidomo/lists/
\end{verbatim}
\end{quote}
If you run the ``install-testlist'' target, the following
directory/files will be created additionally:
\begin{quote}
\begin{verbatim}
/usr/local/var/petidomo/lists/testlist/
/usr/local/var/petidomo/lists/testlist/config
/usr/local/var/petidomo/lists/testlist/acl
/usr/local/var/petidomo/lists/testlist/list
\end{verbatim}
\end{quote}

\section{Configuring Sendmail}

Before you can use Petidomo, you have to configure sendmail so that it
knows about Petidomo --- I assume that you have sendmail installed
already. If you are using an MTA other than sendmail, you are on your
own from here on, I am afraid. Any users who have successfully
installed Petidomo on a qmail-, vmailer-, or postfix-based system are
more than welcome to contribute to this documentation to help other
users.

To run Petidomo via sendmail --- what is what you want to do ---, you
have to create apropriate aliases for it. You can do this by adding
the folling lines to your {\sf aliases} file, which usually resides in
{\sf /etc/aliases} or, with newer sendmail versions, in {\sf
/etc/mail/aliases}:
\begin{quote}
\begin{verbatim}
petidomo-manager:   postmaster
petidomo:           "|/usr/local/bin/petidomo --mode=listserv"
petidomo-approve:   "|/usr/local/bin/petidomo --mode=approve"
\end{verbatim}
\end{quote}
In case you installed the Petidomo binary to some other location, you
will have to change the paths here apropriately of course. You may
also chose that mail for the ``petidomo-manager'' should go to some
different address than ``postmaster'', if that suits your needs
better; the main point is that somebody actually \emph{reads} what
arrives there.

If executed the ``install-testlist'' target earlier and thus have the
example mailing list from the distribution installed, you may also
want to add the lines:
\begin{quote}
\begin{verbatim}
testlist:           "|/usr/local/bin/petidomo --mode=deliver testlist"
testlist-request:   "|/usr/local/bin/petidomo --mode=listserv testlist"
testlist-owner:     petidomo-manager
\end{verbatim}
\end{quote}
Having done all this, execute the {\sf newaliases} utility to rebuild
sendmail's internal database. Your changes will not have any effect
unless you do this.

\section{Configuring the File Permissions}

The final step, before Petidomo is successfully installed, is to set
the right permissions to the installation directories and installed
files. Unfortunately, the installation process can not do this
automatically; you have to chose what permissions are ``right''
yourself. If works like this: Petidomo will be called from sendmail,
thanks to the aliases you just created. That means, that sendmail
determines under what user to start Petidomo. In 99\% of all
configurations I have ever seen, that user is ``daemon'', but it
\emph{may} be something else, so we better figure it out for sure.

Add the line
\begin{quote}
\begin{verbatim}
foobar:             "/tmp/foobar-mail"
\end{verbatim}
\end{quote}
to your aliases file and execute {\sf newaliases}. Then send an e-mail
to the address ``foobar''. The contents of this mail will be stored in
the file {\sf /tmp/foobar-mail} then and we are interested in the user
who owns this file:
\begin{quote}
\begin{verbatim}
root@peti:/# sendmail -v foobar </dev/null
foobar... aliased to "/tmp/foobar-mail"
"/tmp/foobar-mail"... Sent
root@peti:/# ls -l /tmp/foobar-mail
-rw-------   1 daemon   daemon        269 Feb 12 17:57 /tmp/foobar-mail
\end{verbatim}
\end{quote}
See? On my system it is ``daemon'' indeed. On your system it may be
someone else. Now that we know, you may remove the foobar-line from
the aliases file again.

OK, sendmail starts Petidomo under user id ``daemon''. This means that
``daemon'' must have read access to virtually any file in the Petidomo
installation. This is the default, because all files are installed
with read permisson for everybody. Also, all directories allow access
to anybody by default. But ``daemon'' also needs write access to the
``localstatedir'' --- {\sf /usr/local/var/petidomo} per default. You
can ensure this by executing the command:
\begin{quote}
\begin{verbatim}
chown -R daemon /usr/local/var/petidomo
\end{verbatim}
\end{quote}

This is a rather simplistic solution to the permisson problem; you
\emph{can} use much more fine-grained settings if you like to. But I
figured that if you are the kind of person who wants to do things like
this, you won't need an explanation how to do it anyway. Just that
much information for you: Petidomo does not actually write to the
``localstatdir'', but only to the subdirectory {\sf ack-queue} located
in it.

Of course, you do not necessarily need to have the {\sf ack-queue}
directory owned by ``daemon'', you can also set the group permissions
apropriately. Furthermore, Petidomo will usually want to write to the
{\sf lists} directory located in the ``localstatedir'', because most
list administrators tend to place the mailing list archives there, but
you can enable write access according to the list's configuration once
you know how you're mailing lists are configured. In case something
does not work as expected, check out the syslog messages for the {\sf
LOG\_MAIL} facility --- this is where Petidomo logs its error messages.

\section{Configuring Petidomo}

The last step before we can test our installation is to configure
Petidomo. This is really simple. List the contents of the
``sysconfdir'' you chose. If you did not change the default paths,
this is {\sf /usr/local/etc}. There you will find two files: {\sf
petidomo.conf-sample} and {\sf petidomo.acl-sample}. Just rename them
to {\sf petidomo.conf} and {\sf petidomo.acl} respectively and fire up
your favorite text editor to edit the file {\sf petidomo.conf}.

Uncomment the options ``Hostname'', ``AdminPassword'', and ``MTA'' and
set the values correctly. ``Hostname'' should be the fully qualified
domain name of the machine running Petidomo. It is essential that this
name can receive e-mail, that is, that is has an MX record. (Talk to
the person administrating the domain name service of your organization
if this doesn't make any sense to you.) As ``AdminPassword'', you can
chose pretty much any text you like, just make sure you remember it.
The ``MTA'' setting will usually be alright the way it is. You may
want to check whether sendmail does actually live at this path; on
some Unixes, it is not installed at {\sf /usr/sbin/sendmail}, but at
{\sf /usr/lib/sendmail}. Change the setting if this is the case. You
can ignore all other settings right now. Come back and configure those
once you have read the apropriate sections of this manual. If you're
an experienced Unix wizard, the comments in the config file will
probably be enough for you to guess what these options do, though.

Once you have done this, your installation is ready to be tested.

\section{Testing the Installation}
\label{testing}

Asserting you followed all steps described above, you have a working
Petidomo installation now. Occasionally, some minor permisson problem
may still remain to be fixed, or you may want to customize some texts.
To figure out what is left to do (or to realize that there is nothing
left do to), send an e-mail to the ``petidomo'' user on your machine
and put the word ``help'' into the mail body --- without the quotes of
course.

On my system, this looks like this:
\begin{quote}
\begin{verbatim}
simons@peti:~/projects/petidomo$ echo help | sendmail -v petidomo
petidomo... aliased to "|/usr/local/bin/petidomo --mode=listserv"
"|/usr/local/bin/petidomo --mode=listserv"... Connecting to prog...
"|/usr/local/bin/petidomo --mode=listserv"... Sent
\end{verbatim}
\end{quote}

Once you sent the e-mail, sendmail will start up Petidomo and feed the
mail text into it for processing. If you take a look at the syslogfile
containing the {\sf LOG\_MAIL} facility now --- this is usally {\sf
/var/log/messages} or {\sf /var/log/maillog} ---, you will find that
Petidomo logged entries there that look pretty much like the following
ones. The backslash (``\verb#\#'') characters at the end of some of
these lines denote that the line has been wrapped for readability. In
reality, this would be one single large line.
\begin{quote}
\begin{verbatim}
sendmail[8705]: f1CIHWJ08705: from=simons, size=5, class=0, \
        nrcpts=1, msgid=<200102121817.f1CIHWJ08705@peti.cryp.to>, \
        relay=simons@localhost
petidomo[8706]: Petidomo 4.0b1 (18-Jan-2001) starting up; \
        mode=listserv, listname=<none>, \
        masterconf=/usr/local/etc/petidomo.conf, \
        approved=false, ruid=2, euid=2, gid=2, egid=2
petidomo[8706]: simons@peti.cryp.to: help
sendmail[8707]: f1CIHX508707: from=petidomo-manager@peti.cryp.to, \
        size=2091, class=-100, nrcpts=1, \
        msgid=<200102121817.f1CIHX508707@peti.cryp.to>, \
        relay=daemon@localhost
sendmail[8705]: f1CIHWJ08705: \
        to="|/usr/local/bin/petidomo --mode=listserv", \
        ctladdr=petidomo (2/0), delay=00:00:01, xdelay=00:00:01, \
        mailer=prog, pri=30005, dsn=2.0.0, stat=Sent
sendmail[8709]: f1CIHX508707: to=simons@peti.cryp.to, delay=00:00:00, \
        xdelay=00:00:00, mailer=local, pri=212091, dsn=2.0.0, stat=Sent
\end{verbatim}
\end{quote}

As you can see, Petidomo logged how it was started, where it is
expecting its master config file and under which user- and group id it
is running. Then it logs that it has received a HELP request. This
request will be answered by sending the file {\sf
/usr/local/share/petidomo/help} back to the person who requested help,
and if everthing worked, you will now find that mail in your mail box.

If something went wrong, Petidomo will tell you what went wrong. So,
please fix the problem and try again. In 99\% of all cases, the error
will say something like ``opening file XYZ failed: permission
denied''. Then all you have to do is to make sure that the user under
which Petidomo has been started (you have the numeric id in the
logfile) has read access to that file. If the user has but Petidomo
keeps complaining, check, whether that user has access to the
directory at all!

Those of you who executed the ``install-testlist'' target earlier in
the ``Building the Binaries'' chapter may also want to test whether
this mailing list is working. To do so, send another mail to Petidomo
and put the command ``subscribe YOUR-ADDRESS testlist'' in the mail
body --- without the quotes! ``YOUR-ADDRESS'' naturally means that you
should insert your e-mail address here. This command will subscribe
your e-mail address to the ``testlist'' mailing list; you should
receive a confirmation about that via e-mail within moments. Once that
is accomplished, send another e-mail to the ``testlist'' address on
your system. The e-mail may look like whatever you want.

Within seconds, you should get the mail back from the mailing list
server --- and so will all other addresses that are subscribed to the
list. My personal test mail looked like this:

\begin{quote}
\begin{verbatim}
From testlist-owner@peti.cryp.to  Mon Feb 12 19:43:56 2001
Received: (from daemon@localhost)
        by peti.cryp.to id f1CIhuA08872 for simons@peti.cryp.to;
        Mon, 12 Feb 2001 19:43:56 +0100
Received: (from simons@localhost)
        by peti.cryp.to id f1CIhJY08853 for testlist;
        Mon, 12 Feb 2001 19:43:19 +0100
Date: Mon, 12 Feb 2001 19:43:19 +0100
From: Peter Simons <simons@peti.cryp.to>
Message-Id: <200102121843.f1CIhJY08853@peti.cryp.to>
Subject: Petidomo absolutely rules the known earth!
Reply-To: testlist@peti.cryp.to
Sender: testlist-owner@peti.cryp.to
Precedence: list

It does ...
\end{verbatim}
\end{quote}

If this all worked for you, you have a your Petidomo installation up
and running. Men will envy you and women will desire you --- unless,
of course, you \emph{are} a woman, then it is vice versa. You will be
able to stop smoking any time you want, you may eat anything you like
and as much as you like, but you will never gain a single pound. Your
sex life will improve dramatically, your boss will like you, your hard
drives will never crash and your Internet connections will always be
fast. All this, thanks to the wonders of the {\bf Petidomo Mailing
List Manager!}

In case any of the benefits promised above stays away, please consult
paragraphs 11 and 12 of the file {\sf COPYING} included in this
distribution.

\chapter{Configuring Petidomo}

\section{Configuration File Syntax}

All configuration files in the Petidomo-package\label{Config file
format}, have the following format:
\begin{quote}
\begin{verbatim}
keyword         parameter
\end{verbatim}
\end{quote}

The ``keyword''-part must start at the first column of the line and is
followed by one or several blanks or tabs. The first non-blank
character then is interpreted as the parameter for this keyword. The
following line, for example:
\begin{quote}
\begin{verbatim}
Hostname        petidomo.is.great
\end{verbatim}
\end{quote}
will tell Petidomo that the name of the machine it is running on is
called ``petidomo.is.great''. If the parameter contains any blanks,
what is not very likely for a hostname, but may happen with other
settings, you should enclose it in double quotes, like this:
\begin{quote}
\begin{verbatim}
AdminPassword   "open sesame"
\end{verbatim}
\end{quote}

Quoting the parameter is not strictly necessary, though, Petidomo's
config file parser will get it right anyway. You only have to quote
the parameter, if it contains blanks as first or last character, what
is rather unlikely to happen.

Furthermore all empty lines are ignored. So are lines that start with
a `\#' sign. You can use this for writing comments for the reader into
the config file.

\section{The Master Configuration File}
\label{master config file}

The following keywords are recognized in the master config file.

\begin{description}

\item[Hostname] \hfill ``hostname.domain.name''

This entry specifies the fully qualified domain name of the machine,
Petidomo is running on. A fully qualified domain name is the
hostname of the machine with the domain name appended with a dot. The
following, for example:
\begin{quote}
\begin{verbatim}
HostName        listserver.foo.bar
\end{verbatim}
\end{quote}
would be a valid statement. Normally this option has been set by the
install script correctly already.

The name you set here is not necessarily the name, Petidomo will use
when delivering mailing list-postings to the subscribers, or when
answering requests, because you can specify a different fully
qualified domain name for every mailing list you host. This is known
as \Def{virtual hosting}.

This option is \emph{required}. Petidomo will abort with an error,
if the master config file doesn't set it.

\item[AdminPassword] \hfill ``password''

This tag sets the master password, which authenticiates the Petidmo
administrator. Here is an example:
\begin{quote}
\begin{verbatim}
AdminPassword   "open sesame"
\end{verbatim}
\end{quote}
Normally this option has been set by the install script already.

Please chose this password carefully. Knowledge of the master password
will enable you to access \emph{all} mailing lists running on this
system.

Passwords are compared case-insensitively. That means, that the
passwords ``Open SESAME'', ``open sesame'' and ``OPEN seSAme'' are all
the same.

This option is \emph{required}. Petidomo will abort with an error,
if the master config file doesn't set it.


\item[MTA] \hfill ``/path/to/executable''

The MTA tag tells Petidomo which mail transport agent should be used
to deliver outgoing emails. Normally this option has been set by the
install script already, so you don't need to worry about this anymore.

An example setting is:
\begin{quote}
\begin{verbatim}
MTA     "/usr/sbin/sendmail"
\end{verbatim}
\end{quote}
but Petidomo will run fine with other mail transport agents, too. So
far, the system has been tested with the Allman sendmail, SMail and
qmail without any problems.

This option is \emph{required}. Petidomo will abort with an error,
if the master config file doesn't set it.


\item[MTAOptions] \hfill ``string''

This tag is a bit tricky and in ninety-nine out of hundred cases you
should simply leave this option undefined as it is rarely required
anyway.

This entry sets the options which will be handed over to the MTA
when it is called. The following example
\begin{quote}
\begin{verbatim}
MTAOptions "-odq -i -f%s"
\end{verbatim}
\end{quote}
will yield a call ``$<$MTA$>$ -odq -i -f$<$envelope$>$''. The `\%s' is
replaced with the envelope the mail should be sent under.

Adding options to the execution call of the mail transport agent can
be useful to enable or disable certain features for mailing lists
only, while leaving them on for all other mail. The `-odq' setting is
a fine example. This parameter will tell the Allmann sendmail to queue
all mail, instead of trying to deliver it immediately.


\item[ListDirectory] \hfill ``/path/to/directory''

Here you can tell Petidomo the path to the directory where the mailing
list config file reside. The compiled-in default is
``LOCALSTATEDIR/petidomo/lists'' --- in most cases that resolves to
\file{/usr/local/var/petidomo/lists}. When Petidomo tries to find the
configuration of list, say, ``foobar'', it will look for any of the
following files in this directory: \file{foobar.conf},
\file{foobar.config}, \file{foobar/conf}, or \file{foobar/config}.

\item[AckQueueDirectory] \hfill ``/path/to/directory''

This tag will tell Petidomo where to store files that need to be
queued for later processing --- for example subscribe requests that
need to be acknowledged by the user before they'll be carried out. The
default location is ``LOCALSTATEDIR/petidomo/ack-queue''; unless you
changed the default path at compile time, this will resolve to
\file{/usr/local/var/petidomo/ack-queue}. Please note that Petidomo
will need permission to write to that directory in order for things to
work.

\item[HelpFile] \hfill ``/path/to/file''

This tag will tell Petidomo where to find the text file that will be
sent back to a user sending in a ``help'' command. The default
location is ``DATADIR/petidomo/help"'', what is usually the file
\file{/usr/local/share/petidomo/help}.


\item[ACLFile] \hfill ``/path/to/file''

This tag tells Petidomo the path to the system-wide ACL file. The
default location is ``SYSCONFDIR/petidomo.acl'', what usually resolves
to \file{/usr/local/etc/petidomo.acl}.

\item[IndexFile] \hfill ``/path/to/file''

Similarly to ``HelpFile'', this tag will tell Petidomo where to find
the text file that will be sent back to a user requesting the server's
``index''. The default location is ``LOCALSTATEDIR/petidomo/index'';
unless you changed the default path at compile time, this will resolve
to \file{/usr/local/var/petidomo/index}.

\end{description}

\section{List Configuration Files}
\label{list config file}

While the master config file sets options which are relevant for the
Petidomo package as a whole, the list config file sets options which
are valid only locally for the mailing list. The supported keywords
are as follows.

\begin{description}

\item[ListType] \hfill ``open'', ``closed'',  ``moderated'',
``acknowledged'', or ``acknowledged-once''

This setting tells Petidomo who is allowed to post to the mailing
list. On an ``open'' mailing list, everybody is allowed to post,
whether he's subscribed or not. On a ``closed'' mailing list, only
subscribers are allowed to post. That means that only mails coming
from an address found on the list's address database will go through.
All other mails will be sent back to the person trying to post with
the comment that he should subscribe first.

Please note that a closed list may not do exactly what you want,
because when a person is subscribed to a list as
``example@address.net'', but tries to post from a different account
than that one, Petidomo will not let him post. It tries to recognize
this case as far as possible: For example, it doesn't matter whether
you are posting from ``address@host1.address.net'' or
``address@host2.address.net'', Petidomo will handle that. But if the
article comes from ``example@private.account'', it will be rejected,
even though the sender might be a valid subscriber. It depends on the
subscribers of the mailing list, whether this is a problem or not.

A better way to keep spam off your lists may be the ``acknowledged''
or ``acknowledged-once'' type of list. The former list mode means that
every time someone tries to post, he will get a short reply back which
contains some cryptographic cookie. Furthermore the mail will tell
him to please reply to that mail and send the cookie back to Petidomo.
Once that is accomplished, Petidomo will let the posting pass.

This means that only people will be able to post that have a valid
envelope or from address --- something spammers usually do not. Since
the request for confirmation never reaches them, their postings will
not go through. Everybody else can acknoledge the posting by sending
the cookie back and thus get by that hurdle.

The difference between the ``acknowledged'' and ``acknowledged-once''
mode finally is that in the latter mode, people have to send that
confirmation back only \emph{once}, while in the first mode, they have
to send it in every time they're trying to post. Petidomo will keep a
database of addresses that have been confirmed by that mechanism and
these addresses are allowed to post from now on without having to
confirm their posting again. Please note that this list of addresses
is not related to the list of subscribers!

Last but not least, there is a mode called ``moderated''; in this
mode, nobody is allowed to post unless he can provide the correct
posting- or administrator password for the list.

If this option is unset, the default is to run an open list.

\item[SubscriptionType] \hfill ``public'', ``admin'', or ``acknowledged''

This option specifies who may subscribe to a mailing list. On an
``open''-subscription list, anybody may subscribe any address to the
list. If the subscription type is ``admin'', only the person knowing
the admin password of the mailing list is allowed to subscribe or
unsubscribe addresses. All other requests that aren't properly
authenticated will be forwarded to the list owner for approval.

If the subscription type is set to ``acknowledged'', anybody can issue
a subscribe command, but the address will not be added to the list (or
removed from the list in case of an ``unsubscribe'') unless the
request has been confirmed: Petidomo will send a random cookie to the
address in questing and ask to send the cookie back to acknowledge the
request. Once the cookie has been send back, the request is carried
out.

This mode is useful to prevent people from adding random addresses to
the list or to prevent them from removing other people from the list
without their consesus.

If this option is unset, the default to allow public subscription.

\item[AllowMembersCommand] \hfill ``yes'' or ``no''

Petidomo knows a command ``members'' or ``who'', which can be sent
to the server and it will reply with the complete list of subscribed
addresses for the mailing list. This may be useful for list
administrators, but it can be abused easily by spammers, to collect
addresses where to send their unsolicted commercial e-mail to.

Furthermore, with certain mailing lists it may be undesirable that one
can see ``who else'' is subscribed to that list. That's why this
option has been added. If you set it to ``no'', the
``members''-command will be diabled for this list. (This is also the
default if the option is not specified in the config file.)

If you set it to ``yes'', the ``members''-command will work.

\item[Hostname] \hfill ``hostname.domainname''

This options tells Petidomo to use this hostname for the mailing list,
instead of the one configured in the master configuration file. This
feature is useful to do virtual hosting.

\Def{Virtual hosting} is required when several mailing lists run on
the same server, but they have to look like they would coming from
different machines. Let's use an example: The internet service
provider ``Inter.Net'' offers its customers to host mailing lists for
them. A small software house of the name ``Petiware'' wants to provide
a mailing list for all its customers, but they don't have a dedicated
Internet line.

So they use the service provided by Inter.Net and let them host the
mailing list on their machine. The mailing list server at Inter.Net
has the fully qualified domain name ``mail.inter.net''. Petiware,
though, wants the list to run under the name
``customers@petiware.com'' and \emph{not} ``customers@inter.net'' ---
what would a be misleading.

So all the mailing list guru from Inter.Net has to do is to set the
entry
\begin{quote}
\begin{verbatim}
Hostname        petiware.com
\end{verbatim}
\end{quote}
in the config file of the ``customers'' mailing list. Petidomo will
now use the hostname ``peti\-ware.com'' in all mails that are posted
to that list, instead of ``mail.inter.net''.

You can specify a different hostname for every mailing list, using
this feature. \emph{That} is ``virtual hosting''. Further details on
virtual hosting can be found in section~\ref{virtual hosting and
sendmail} of the user manual.

If this entry is unset, the name configured in the master config file
will be used as hostname for this mailing list.

\item[AdminPassword] \hfill ``string''
\label{list admin password}

This tag sets the master password, which authenticiates the
administrator of this mailing list. The administrator has special
priviledes, such as deleting other users, overriding access control
restrictions or un-/subscribing users to closed mailing lists. This is
described briefly in section~\ref{petidomo as admin} of the user manual.

Please note that passwords are always case-insensitive. It is also
worth noting that the master password is always valid as administrator
password for the list, also.

Leave this entry blank, if you don't want to enable remote
administration of the mailing list.

\item[PostingPassword] \hfill ``string''
\label{posting password}

This tag sets the ``posting password''. The posting password allows to
post an article to a moderated mailing list, but it does not allow any
administration of the list itself. On lists that are of a different
type than moderated, setting a posting password does usually not make
any sense and you can leave this entry unset.

\item[ReplyTo] \hfill ``email@address.net'' or ``none''

This tag controls the `Reply-To:' field, which Petidomo adds to
posted articles before it is delivered to the recipients. Using this
option, you can force Petidomo to insert a `Reply-To:' which points
to a certain address. On a moderated list, for example, you can set
this as follows:
\begin{quote}
\begin{verbatim}
ReplyTo         moderator@address.net
\end{verbatim}
\end{quote}
to direct all replies to the posting to the moderator again,
regardless of what address is noted in the `From:' line of the mail.

If you set ``none'', Petidomo will not add a `Reply-To:' header at
all.

If this option is unset, Petidomo will to insert a `Reply-To:'
header that directs replies back to the mailing list, so that
subscribers can conveniently post simply by hitting the `reply'
function in their mail reader.

\item[PostingFilter] \hfill ``bourne shell command''

If you specify a posting filter, this program or script will be
started by Petidomo before it sends a posting out to the
subscribers. The programm will receive the article, as it has been
prepared by Petidomo, on standard input and is expected to write the
final version of the mail to standard output. The posting filter can
be used to manipulate the headers for special purposes.

An example for a postin filter that wouldn't modify the mail at all is
the following:
\begin{quote}
\begin{verbatim}
PostingFilter   /bin/cat
\end{verbatim}
\end{quote}

A detailed discussion of posting filters can be found in
section~\ref{using posting filters} of the manual.

If the filter program exits with a returncode not equal to 0 (zero),
Petidomo will not post the article and terminate.


\item[Archive] \hfill ``/path/of/archive''

If this option is set, Petidomo will archive all articles that have
been posted on that mailing list. The parameter for this tag may
either be the name and path of a file or of a directory. The path may
either be absolute (\file{/var/archive/list}) or relative
(\file{archive}). For relative paths, the directory where the list's
config file resides will be used as starting point. If the
``Archive''-tag points to a file, Petidomo will append every posted
article to that file. If points to a directory, each posting will be
stored in that directory in a separate file.

If this option is unset, posted articles will not be archived at all.

For further information an creating mailing list archives, please
refer to section~\ref{mailing list archives} of the user manual.


\item[IntroductionFile] \hfill ``/path/to/file''

This tag specifies the path to the so called ``introduction'' file.
Every time an address is added to the mailing list, Petidomo will send
the contents of this file to the new subscriber. This is meant to be
used to inform the new subscriber about the list's topic, habits he
should know, etc. If the file does not exist, no mail is sent out.

If the path specified here is relative --- not starting with a ``/''
character that is ---, it is interpreted to be relative to the
directory where the list's config file has been found. The default
path is \file{introduction}.


\item[DescriptionFile] \hfill ``/path/to/file''

This tag specifies the path to the so called ``description'' file.
This file is supposed to contain a short description of the mailing
list's topic and purpose. It's contents will be sent back if a user
requests the command ``help listname''.

If the path specified here is relative --- not starting with a ``/''
character that is ---, it is interpreted to be relative to the
directory where the list's config file has been found. The default
path is \file{description}.


\item[ACLFile] \hfill ``/path/to/file''

This tag specifies the path to the list-specific ACL file. Please
refer to section~\ref{acl} for more information about the access
control language of Petidomo.

If the path specified here is relative --- not starting with a ``/''
character that is ---, it is interpreted to be relative to the
directory where the list's config file has been found. The default
path is \file{acl}.


\item[HeaderFile] \hfill ``/path/to/file''

The contents of this file this tag points to will be added to the
header of \emph{every} posting on this list. This may be used to add
custom headers like:
\begin{quote}
\begin{verbatim}
X-List-Archive-is-at: http://list-archive.example.org/
\end{verbatim}
\end{quote}

Please note that the contents of this file will be added
\emph{verbatim}! So don't include any empty lines in here as empty
lines mark the end of the mail headers! Generally, please use this
feature with care; most mailing list administrators tend to
overestimate the importance of custom headers on their mailing list.

If the path specified here is relative --- not starting with a ``/''
character that is ---, it is interpreted to be relative to the
directory where the list's config file has been found. The default
path is \file{header}.


\item[SignatureFile] \hfill ``/path/to/file''

The contents of this file this tag points to will be appended to
\emph{every} posting on this list. This may be used to add a
list-specific signature, like:
\begin{quote}
\begin{verbatim}
--
 Useful comment here.
\end{verbatim}
\end{quote}

If the path specified here is relative --- not starting with a ``/''
character that is ---, it is interpreted to be relative to the
directory where the list's config file has been found. The default
path is \file{signature}.


\item[AddressFile] \hfill ``/path/to/file''

This tag specifies the path to the path of the file Petidomo uses to
store the list of subscribed addresses. If the path specified here is
relative --- not starting with a ``/'' character that is ---, it is
interpreted to be relative to the directory where the list's config
file has been found. The default path is \file{list}.


\item[AcknowledgementFile] \hfill ``/path/to/file''

This tag specifies the path to the path of the file Petidomo uses to
store the list of addresses that have been verified by the
acknowledgement mechanism and may furtheron post without having to
acknowledge their posting again. This is only used if the list is set
to ``acknowledge-once'' mode.

If the path specified here is relative --- not starting with a ``/''
character that is ---, it is interpreted to be relative to the
directory where the list's config file has been found. The default
path is \file{ack}.

\end{description}

\section{Command Line Syntax}

Petidomo understans several command line parameters. Here is the
complete list:

\begin{description}

\item[{-}-mode={\sf mode}] \hfill ``listserv'', ``deliver'', ``approve'', or ``dump''

The mode parameter is the only mandatory parameter and it will
determine what mode Petidomo runs in. Anyway, if Petidomo is started
in ``listserv'' mode, it will expect to read an e-mail from standard
input, which contains commands like ``subscribe'', ``index'' or
``help''. These commands will be carried out and notifications be sent
back to the mail's originator if apropriate.

In ``deliver'' mode, Petidomo will read the incoming e-mail from
standard input and post it to the mailing list, which's name has been
provided via the ``listname'' option. When running in ``approve''
mode, Petidomo will read the incoming mail from standard input and
search for any cookies that mail might contain. If it does, it checks
the ack-queue for a mail that has been deferred until confirmation
that matches that cookie and processes the mail.

In ``dump'' mode, Petidomo will expect the name of a mailing list on
the command line --- the ``listname'' option --- and dump the list of
subscribed addresses on that list to standart output.

\item[{-}-listname={\sf list name}]

This parameter may contain any valid mailing list name. Depending on
the mode, it this list name will be used as follows. In ``listserv''
mode, that list will be used as default list name for any command
where no list name has been specified. A ``subscribe'', for example''
will subscribe the user to the list given here; a ``subscribe
some-other-name'' will still subscribe the user to that other list,
though.

When running in ``deliver'' mode, this is the name of the list the
posting is supposed to be posted to. In ``dump'' mode, this is the
name of the list, which's subscriber list should be dumped. In
``approve'' mode, this parameter is ignored.

\item[{-}-masterconf={\sf /path/to/petidomo.conf}]

Using this parameter you can tell Petidomo to use a different location
for the master config file than the one that has been compiled in.

\item[{-}-approved]

This flag is for internal purposes and should not be specified by the
administrator. It will tell Petidomo that, whatever it is supposed to
do now, is the result of a received confirmation cookie. This will
effectively tell the checks for posting (in ListType ``acknowledged''
and ``acknowledged-once'' mode) and un-/subscription (in
SubscriptionType ``acknowledged'' mode) that everything is fine and
that the request should be carried out.

\end{description}


\section{Aliases}
\label{aliases}

The Petidomo binary will usually not be called manually from the
shell, but by the mail transport agent. This works as follows: You
create an e-mail account, which serves the purpose of accepting the
incoming e-mail and piping it into the Petidomo binary.

This is archieved with the ``alias''-function of your mail transport
agent. Most MTAs, like sendmail, have a file where a list of special
account names is given together with the instructions what to do with
any mail received for that account. This file is usually located at
\file{/etc/aliases} or \file{/etc/mail/aliases}.

One thing, aliases can do is pipe the mail into a program for
processing. This is the mechanism Petidomo uses. Petidomo requires you
to add the following aliases to your system:
\begin{quote}
\begin{verbatim}
#
# Mailing List Stuff
#
petidomo-manager:   postmaster
petidomo:           "|/usr/local/bin/petidomo --mode=listserv"
petidomo-approve:   "|/usr/local/bin/petidomo --mode=approve"
\end{verbatim}
\end{quote}

The lines starting with the `\#' character are only comments and are
ignored by the mail transport agent. The fourth line, though, is the
first command. It tells the MTA to accept mail for an user of the name
``petidomo-manager'' and to re-direct the e-mail to the address
``postmaster'' --- the mail system administrator.

Petidomo will send error notifications and things like that to the
address ``petidomo-manager''. By setting this alias to a certain user
name, you can control who will receive those mails.

The next line tells the MTA to pipe any incoming mail for the user
``petidomo'' into the ``petidomo'' program, instead of delivering it
into a mailbox. ``petidomo'' (in listserv mode) will then parse the
mail for commands and react accordingly. Hence, the address people can
send their subscription requests to is ``petidomo@your.host.name''.

Similarly, the address ``petidomo-approve'' will be used to receive
any acknowledges user send back after Petidomo requested them. Only
now, Petidomo is started in ``approve'' mode.

Furthermore, each mailing list on your server \emph{requires} three
aliases, as shown in the example below, which is written for the
``testlist'' mailing list that comes with the distribution:
\begin{quote}
\begin{verbatim}
testlist:           "|/usr/local/bin/petidomo --mode=deliver testlist"
testlist-request:   "|/usr/local/bin/petidomo --mode=listserv testlist"
testlist-owner:     petidomo-manager
\end{verbatim}
\end{quote}

The first alias, ``testlist'' is the address to which people can send
their mail in order to post to the mailing list. Any incoming mail for
that account will be piped into the ``petidomo'' binary in ``deliver''
mode, which will process the mail and then re-send it to all
subscribers of the mailing list. In order to let Petidomo know, for
which mailing list the posting was meant, the parameter ``testlist''
has to be specified on the command line. If the name of the mailing
list was ``foobar'', the line would look like this:
\begin{quote}
\begin{verbatim}
foobar:             "|/usr/local/bin/petidomo --mode=deliver foobar"
\end{verbatim}
\end{quote}

The second alias is a special request address, to which users can send
their commands. The difference between this address and the
``petidomo'' alias described above is that here Petidomo is being
given a default listname on the command line. The difference is this:
If Petidomo receives a mail, which has the command ``subscribe'' in
it, without any further parameters, it will reject the command with an
error, because it doesn't know to which list the sender wants to be
added.

If the command ``subscribe'' is sent to the ``testlist-request''
address, though, it will assume that the user wants to be subscribed
to the ``testlist'' mailing list, as this is the default list for this
address.

The name of this alias should always be the name of the mailing list
with the string ``-request'' appended. Theoretically you could choose
a different name, but this unwritten standard has been widely accepted
in the Internet for several years now.

The last alias is the name of the mailing list with the string
``-owner'' appended. This alias points to the person who is
responsible for managing the ``testlist'' mailing list. Petidomo
will send all e-mail concerning the administration of the mailing list
to the address ``listname-owner''. Usually this will ultimately be the
same person as the ``petidomo-manager'', but you are free to direct
mail for this account to somebody else, or to several persons.

\chapter{Petidomo for Mailing List Users}
\label{petidomo as user}

In this chapter, we will describe the commands, that are understood in
``listserv'' mode. This is the interface for the users of the mailing
lists, where they can send their requests to in order to be subscribed
to a mailing list, be unsubscribed again and similar things. The text
here is mostly identical with the default help text that is sent to
the user whenever he or she issues a command that is syntactically
incorrect.

User commands always have to be sent to the request address of
the mailing list --- \emph{not} to the mailing list itself!

Alternatively, commands can always be sent to the address
``peti\-do\-mo@your.ad\-dress'', but the ``-request''-address is
preferable, for the fact that the Petidomo will have a default
listname for this address and thus understand a simpler command
syntax.

\section{SUBSCRIBE}

The ``subscribe'' command will add the address of the user to a
mailing list. When using the ``-request''-address, only the word
``subscribe'' is required for the request to suceed. If the command is
sent to the ``petidomo'' address, the user will have to specify an
additional parameter: The name of the mailing list he or she wants to
be added to, like in the following example:
\begin{quote}
\begin{verbatim}
subscribe politics
\end{verbatim}
\end{quote}

If the user wants to add an address that is not equal to the one he or
she is sending the e-mail from, the e-mail address will have to be
specified, too:
\begin{quote}
\begin{verbatim}
subscribe politics joe@foo.bar
\end{verbatim}
\end{quote}

The order in which the e-mail address and the mailing list name are
provided does not matter. Please note that the administrator can
configure Petidomo to disallow un-/subscring other addresses than
the one, the request is sent from, using the
``AllowAlienSubscription'' option in the list's config file.

The command ``add'' is synonymous to ``subscribe''.

\section{UNSUBSCRIBE}

The syntax and usage of the ``unsubscribe`` command are the same as the
``subscribe'' command. The difference is, though, the the user's address
is removed from the mailing list rather than added to it.

``delete'' and ``remove'' can be used synonymously to ``unsubscribe''.

\section{INDEX}

The ``index'' command does not need any parameters. Sending it to the
server will return a list of available mailing lists on this server.
This is useful in case you want to subscribe to a list but can't
remember the exact name anymore.

The commands ``lists'' and ``longindex'' are synonyms to ``index''.

\section{HELP}

If the server receives the command ``help'', it will send the help
file back. If ``help'' has a parameter, Petidomo will check whether
this is a valid name of an existing mailing list, and if it is, it
will return the description file for this mailing list, rather than
the help file.

\section{MEMBERS}

The ``members'' command will return the addresses of all subscribers
of the mailing list, if the administrator chose to allow this command.
When ``members' is sent to the ``-request''-address, the default list
will be used by Petidomo. Otherwise, the name of the mailing list
which's subscribers should be listed, has to be specified as an option
like in the following example:
\begin{quote}
\begin{verbatim}
members politics
\end{verbatim}
\end{quote}

The command ``who'' can be used synonymously to ``members''.

\chapter{Petidomo for Administrators}
\label{petidomo as admin}

On the ``other side'' of Petidomo, from the user's
perspective, is the administrator of the mailing list --- also called
the \Def{mailing list owner}). Each mailing list has an
alias ``listname-owner'' (see section~\ref{aliases}), where the mail
address of the person who is responsible for this mailing list should
be specified. Per default, this is the user who is known as
``petidomo-manager''. But you are free to direct mail for this accoun
to any other person --- or several persons.

The list owner will receive administrative e-mail from Petidomo in
the following cases:

\begin{itemize}

\item When a new user subscribes, or a subscriber removes himself from
the list, a carbon copy of the recipt will be sent to the owner. By
looking at these mails, the owner can check whether a ``subscribe'' or
``unsubscribe'' command looks bogus. He or she can also keep track of
who is on the list and who is not.

\item If a ``members'' command is received for a mailing list where
this command has been disabled, this will also be forwarded to the
owner.

\end{itemize}

These mails are merely for information purposes and do not necessarily
require an action from the admin. There are cases, where the list
owner will receive mails from Petidomo, though, that require some
kind of reaction.

\section{Bounces}

While maintaining mailing list with a larger number of subscribers, it
happens regularly that subscribed addresses become invalid or are
temporarily not reachable. In this case postings will \Def{bounce}.
You will then receive a mail from a mail server telling you, that the
delivery of the mail failed.

Often, addresses become unreachable due to a misconfiguration of a
machine, so it is not always necessary to remove that address from the
list immediately, but when an addresses bounces for several days in a
row, it is a good idea to delete that address from the mailing list.
You should do that by sending an ``unsubscribe'' command for that
address to the ``-request''-address of the mailing list.

If you have configured Petidomo to disallow the unsubscription of
addresses not equal to the address the mail is sent from, you will
have to specify your admin password in the mail, to override the
barrier. How this is done is described in section~\ref{approve} later.

\section{Closed and moderated lists}

If you have configured a mailing list to reject postings under certain
circumstances, such as a closed or moderated mailing list, these
rejected articles will be forwarded to you for approval. When you
receive such a rejected article, you can either silently
discard it, contact the author or post it to the mailing list with
your approval.

You can approve an article with the master password for Petidomo, the
admin password of the mailing list in question or the posting password
(see section~\ref{posting password} of that list.

\section{Approving requests}
\label{approve}

To approve an article, you have several ways of specifying the
appropriate password. They are all the same for Petidomo and it is
only a matter of taste, which scheme you use.

When sending a command to Petidomo in ``listserv'' mode through the
``-request'' or ``petidomo''-address, it is easy: Just preface your
commands with a ``password'' command, like in the following example.
\begin{quote}
\begin{verbatim}
To: testlist-request@foo.bar
Subject:

password open sesame
subscribe some@one.else
subscribe someone@even.elser
\end{verbatim}
\end{quote}

One ``password'' command sets your password for all the commands to
follow. If you want to use one mail to send requests for several
mailing lists with different passwords, just give a ``password''
command again:
\begin{quote}
\begin{verbatim}
To: petidomo@foo.bar
Subject:

password open sesame
subscribe user@inter.net testlist1
password let me in
subscribe user@inter.net testlist2
\end{verbatim}
\end{quote}

Instead of ``password'', you can also use the commands ``passwd'', or
``approve'', they are all synonymous.

\section{Approving postings}

If you want to approve a posting for a mailing list, just send the
article to the mailing list and specify your password either in the
header or in the body of the mail.

If you choose to approve the mail in the body, add line with the
command ``approve'' to the mail as first line of the body. Petidomo
will strip that line before actually posting the article then. You can
also use the synonyms ``approved'', ``password'' or ``passwd''
instead. Here is an example:
\begin{quote}
\begin{verbatim}
From: simons@computer.org (Peter Simons)
Subject: Cats are the most beautiful animals in the world.

approve let me post
It's not that I wouldn't like animals like dogs, birds
or fishes, but for me, a cat is *the* animal to have.
[...]
\end{verbatim}
\end{quote}

The line ``approve let me post'' will be stripped by Petidomo and
then the article will be sent out.

If you want to specify the password in the headers, just add an header
of the name ``Approved'' or ``Approve'' to the headers of the mail.
(Unfortunately, many mail readers do not allow you to modify the
headers of outgoing mail. That is why the body-approval has been
added.) Here is the same example as above now using the headers:
\begin{quote}
\begin{verbatim}
From: simons@computer.org (Peter Simons)
Subject: Cats are the most beautiful animals in the world.
Approve: let me post

It's not that I wouldn't like animals like dogs, birds
or fishes, but for me, a cat is *the* animal to have.
[...]
\end{verbatim}
\end{quote}

Please note that you have to add a colon to the keyword to make a
valid RFC mail-header.


\section{The Access Control Language}
\label{acl}

Unfortunately, we live in a world where some people are trying to
abuse services like mailing lists for their entertainment or for
commercial purposes. It is also not uncommon that among thousands of
mailing list subscribers, there is one particular moron who simply
can't behave. That is why access control is a useful feature, even
though it contradicts the idea of a mailing list: To be a media for
communication.

Writing and understanding ACL files is, to be honest, not very easy
and the novice mailing list administrator should better be careful
when using them, because a wrong access control rule might cause more
trouble than it is worth, but the experienced administrator will
certainly appreciate their power. Understanding how ACL files work
will also require you to know a bit about the syntax of an RFC format
e-mail. A good place to start is to take a look at RFC822 and its
sons.

In Petidomo, two places exist to control who is allowed to do what:
The global acl file and the acl file that is local to the mailing
list. While the latter is valid only for the list in which's home
directory it is stored, the globl acl file will be parsed for
\emph{all} your mailing lists. ACL files are only relevant for mailing
list postings, Petidomo does not use them in ``listserv'' mode.

The syntax of an ACL file is similar to the C programming
language, as you can see in the following example:
\begin{quote}
\begin{verbatim}
if (envelope matches "mailer-daemon@") then
        forward "petidomo-manager";
\end{verbatim}
\end{quote}

This is a simple version of the default ACL file which comes with the
Petidomo distribution. It tells Petidomo to forward all postings to a
mailing list, where the envelope of the mail matches the regular
expression ``mailer-daemon@''. This rule is included in the default
distribution to make sure that bounces of articles will not be posted
to the list again, thus causing an infinite mail loop. The syntax of
an ACL statement is shown in figure~\ref{acl syntax}.

\begin{figure}[bth]
\begin{center}
\begin{tabular}{cccccccccc}
IF & ( & from & match & {\tt "}regexp{\tt "} & ) & THEN & pass & & ; \\
   &   & subject & matches &                   &   &      & drop & & \\
   &   & envelope & ==     & {\tt "}string{\tt "}          &   &      & reject & & \\
   &   & header   & =     &                & &     & rejectwith & {\tt "}file{\tt "}  & \\
   &   & body     &       &                & &     & redirect   & {\tt "}address{\tt "} & \\
   &   &      &       &                &   &      & forward    & {\tt "}address{\tt "} & \\
   &   &      &       &                &   &      & filter     & {\tt "}script{\tt "} & \\
   &   &      &       &                &   &      & approve     &  & \\
IF & ( &  & {\tt "}filter{\tt "} &  & ) & THEN &  & & ; \\
\end{tabular}
\caption{Access Control Language syntax}
\label{acl syntax}
\end{center}
\end{figure}

Admittedly, the figure is rather impossible to understand without
further explaination, don't worry if things are still a bit unclear
after looking at it. There is also an EBNF grammar of the ACL to be
found in figure~\ref{ebnf}, which might help those who can read BNF
much more than the other figure.

\begin{figure}[bth]
\begin{quote}
\begin{verbatim}
input:     /* empty */
         | input statmt
         ;

statmt:    `;'
         | `if' exp `then' action `;'
         ;

exp:       qualifier `=' string
         | qualifier `match' string
         | string
         | exp `or' exp
         | exp `and' exp
         | `!' exp
         | `(' exp `)'
         ;

qualifier: `from'
         | `subject'
         | `envelope'
         | `header'
         | `body'
         ;

action:    `pass'
         | `drop'
         | `approve'
         | `reject'
         | `rejectwith' string
         | `redirect'   string
         | `forward'    string
         | `filter'     string
         ;

string:    `"' [^"]* `"'
\end{verbatim}
\end{quote}
\caption{EBNF of the Access Control Language}
\label{ebnf}
\end{figure}

Every ACL statement looks like this: ``IF condition THEN action ;''.
The condition may or may not be enclosed in brackets. Several
conditions can be combined with the keywords ``OR'' and ``AND''.
Furthermore every condition can prefaced with a ``NOT'', which will
reverse the outcome of the condition.

Let's explain this all at a concrete example: You want to reject all
postings which come from the addresses ``moron@moron.net'' and
``spam@spam.net'', because these people have constantly been abusing
your mailing list service. This can be done with the following two
statements:
\begin{quote}
\begin{verbatim}
IF from == "moron@moron.net" THEN reject;
IF from == "spam@spam.net" THEN reject;
\end{verbatim}
\end{quote}

Using the ``OR'' statement you can combine this into one statement:
\begin{quote}
\begin{verbatim}
IF from == "moron@moron.net" OR
   from == "spam@spam.net" THEN
      reject;
\end{verbatim}
\end{quote}

And now we include brackets for readability:
\begin{quote}
\begin{verbatim}
IF (from == "moron@moron.net") OR
   (from == "spam@spam.net") THEN
      reject;
\end{verbatim}
\end{quote}

The keyword ``from'' stands for the address, noted in the ``From:''
header line of the mail and, the ``== {\tt "}address{\tt "}'' means
that the condition if this address is equal to the one written in
quotes thereafter. (You can also use a single `=' character, if you
prefer that over two equal-characters.) This is a verbatim match. If
we'd use the ``match'' or ``matches'' keyword instead of the ``=='',
the parameter would be interpreted as an extended regular expression
and the condition would be true if the addresses matched this pattern.
(Regular expressions are described in the re\_format(7) man page, or
in the manpages of sed(1), grep(1) or egrep(1).)

Other keywords than ``from'' for the first part of the conditional are
``subject'' (the contents of the ``Subject:'' header), ``envelope''
(the envelope of the mail), header and body. The latter two represent
the whole header or body of the mail and should be used only for
regular expression matches and not for verbatim matches.

A short comment on the difference between ``redirect'' and
``forward'': The ``redirect'' action will send the mail to the
specified address without changing anythin in the mail. All the
headers are left untouched and thus the mail will look as if it has
been sent by the person to that address right away. This is useful for
redirecting mails to daemons or programs, but it will usually confuse
a human recipient

The ``forward'' action, though, will send a mail to the specified
address with a new set of headers, which identify the Petidomo Mailing List Manager as
originator and then it will quote the mail that has been forwarded in
the mail body.

Valid actions are ``pass'' (post the mail immediately), ``drop''
(discard the mail without further notice), ``reject'' (send a mail to
the poster, telling him his posting was rejected), ``rejectwith''
(sending mail to the poster, too, but with the contents of a specified
file), ``redirect'' (redirect the mail to a specified address),
``forward'' (like ``redirect'' but preface the mail with a note
telling why the mail was re-sent) or ``filter'' (pipe the mail into
the specified filter script and post the mail as the filter writes it
to the standard output). Furthermore, there is the ``approve'' action
that allows you to approve the posting, thus bypassing all other
checks.

Here are a few more examples in the hope that they make this all
easier to understand: Let's assume you would like to catch all
postings to your mailing lists, that contain the words ``MAKE MONEY
FAST'' in the subject. Then one way of doing this is the following
statement:
\begin{quote}
\begin{verbatim}
IF (subejct matches "make money fast") THEN
      rejectwith "/usr/local/share/petidomo/make-money-fast.txt";
\end{verbatim}
\end{quote}
The file \file{make-money-fast.txt} could, for example, contain the
following text:
\begin{quote}
\begin{verbatim}
Dear poster,

your mail has been rejected. Please note that chain letters
like the "make money fast" text you tried to post are
illegal throughout the world and your are likely to get in
trouble if you continue to spread them.
\end{verbatim}
\end{quote}

If someone tried to post the chain letter to your mailing lists now,
he would receive a mail like that:
\begin{quote}
\begin{verbatim}
Date: Sat, 28 Jun 1997 19:59:18 +0200 (MET DST)
From: testlist-owner@example.org (Petidomo Mailing List Server)
To: simons@example.org
Cc: testlist-owner@example.org
Subject: Your posting to list "testlist" was rejected
Precedence: junk
Sender: testlist-owner@example.org

Dear poster,

your mail has been rejected. Please note that chain letters
like the ``make money fast'' text you tried to post are
illegal throughout the world and your are likely to get in
trouble if you continue to spread them.

>From simons  Sat Jun 28 19:59:17 1997
Received: from [[UNIX: localhost]]
        by example.org (8.8.5/8.8.4) id TAA16959
Date: Sat, 28 Jun 1997 19:59:17 +0200 (MET DST)
Message-Id: <199706281759.TAA16959@example.org>
From: Peter Simons <simons@example.org>
To: testlist
Subject: MAKE MONEY FAST
Mime-Version: 1.0 (generated by tm-edit 7.92)
Content-Type: text/plain; charset=US-ASCII

Hi, my name is David Rodes...
\end{verbatim}
\end{quote}

A few more words about how the ACL files are parsed:
\begin{itemize}

\item All comparisons are done case insensitive. ``MAKE MONEY FAST''
matches ``make money fast'' in both the verbatim and the regular
expression match just fine.

\item Any whitespace in the ACL file is ignored. The statements
\begin{quote}
\begin{verbatim}
if (envelope matches "mailer-daemon@") then drop;
\end{verbatim}
\end{quote}
and
\begin{quote}
\begin{verbatim}
if
    (envelope matches
"mailer-daemon@")
then
       drop
;
\end{verbatim}
\end{quote}
are the same to Petidomo.

\item The argument after the ``=='' or ``matches'' keyword \emph{has}
to be included in quotes. An ACL statement like this:
\begin{quote}
\begin{verbatim}
if from == simons@computer.org then drop;
\end{verbatim}
\end{quote}
will cause Petidomo to abort with an error, because it can't parse
this.

\item If you use an action that requires a parameter, like
``rejectwith'' or ``forward'', this parameter has to be enclosed in
quotes, too. A statement like this can also not be parsed by
Petidomo:
\begin{quote}
\begin{verbatim}
if from == "simons@computer.org" then
        forward postmaster@example.org;
\end{verbatim}
\end{quote}

\item Petidomo stops parsing the ACL file after the first statement
has matched. If you want to reject all mails from an address that
matches ``simons@.*\.de'', but you want mails from the address
``simons@rhein.de'' to pass nonetheless, the following two statements
will not work as expected:
\begin{quote}
\begin{verbatim}
if from matches "simons@.*\.de" then reject;
if from == "simons@rhein.de" then pass;
\end{verbatim}
\end{quote}

Instead you should use
\begin{quote}
\begin{verbatim}
if from == "simons@rhein.de" then pass;
if from matches "simons@.*\.de" then reject;
\end{verbatim}
\end{quote}
or
\begin{quote}
\begin{verbatim}
if (from matches "simons@.*\.de") and
   (not (from == "simons@rhein.de")) then
         reject;
\end{verbatim}
\end{quote}

\item Currently you can't match for the double quote character ({\tt
"}), we're afraid. The escape sequence {\tt \verb+\+"} is not
supported yet.

\end{itemize}

One last example and then we'll come to the filters. The following
statement rejectes a mail based on a match in the headers. This is
very useful for rejecting mail from known spam domains. You usually
can't rely on the spammer to use a valid ``From:'' header and hence
the ``from''-match is useless to catch them. But the following
statement will usually get them nonetheless:
\begin{quote}
\begin{verbatim}
if (header matches "^Received:.*from spam.domain") then
     forward "petidomo-manager";
\end{verbatim}
\end{quote}

If you thought, the Access Control Language is powerful so far, take a
look at the things you can do using filters. Rather than the examples
described above, you could use the following statement:
\begin{quote}
\begin{verbatim}
if ("/usr/local/libexec/petidomo/CheckPosting") then reject;
\end{verbatim}
\end{quote}

This is a special form of the usual ACL statements and it means the
following: The mail in question is piped into the ``CheckPosting''
script. The script or program can perform various tests and when it
exists, the action part is executed depending on the return code the
script exited with. A return code of zero (0) means ``true'' and the
action will be executed. A return code of one (1) ``false'' and the
action will not be executed.

Any other return code will cause Petidomo to abort with an error and
to save the mail. By using this mechanism, you can program even the
most sophisticated tests and hook them into the access control
mechanism.

Another feature that hasn't been described yet is the action
``filter''. The filter-action is pretty much the same as the posting
filter, but it allows you to re-write the posting depending on who
posted it or other criteria. Please note that this filter is executed
additionally to a regular posting filter you might have configured.

A nice example for what this feature can be used is the following:
\begin{quote}
\begin{verbatim}
if (address == "simons@computer.org") then
       filter "/usr/local/libexec/petidomo/simons.filter";
\end{verbatim}
\end{quote}

The script \file{simons.filter} would then look like this:
\begin{quote}
\begin{verbatim}
#! /bin/sh

cat
echo
echo "-- "
echo " Hold your breath -- this is *the* Peter Simons!"
\end{verbatim}
\end{quote}

We resisted the temptation of adding this ACL statement into the
default configuration of Petidomo.

% \chapter{Administrating Mailing Lists}
%
\chapter{Miscellaneous Topics}
\section{Using posting filters}
\label{using posting filters}

The posting filter functionality of Petidomo is a very useful
mechanism for you, if you run mailing lists where you want to
guarantee certain formal criteria, because you can hook a script or
program of your into the posting process and use it to re-format or
re-write the article that is going to be posted.

We have included one script into the distribution,
\file{Insert\-Name\-In\-Sub\-ject.sh}, which adds a string into the
subject line of every posting. The script is pretty short and used
sed(1) to perform its function.

To use it, just add the line
\begin{quote}
\begin{verbatim}
PostingFilter     ~petidomo/bin/InsertNameInSubject.sh listname
\end{verbatim}
\end{quote}
with ``listname'' being the name of the mailing list.

If the mailing list name was ``testlist'', for example, then this
posting filter would re-write the subject line
\begin{quote}
\begin{verbatim}
Subject: Hi everbody
\end{verbatim}
\end{quote}
to
\begin{quote}
\begin{verbatim}
Subject: [testlist] Hi everbody
\end{verbatim}
\end{quote}

It is recommended to take a look at the script itself to understand
how this works. You will need a bit of knowledge of the Unix scripting
language and tools like sed(1) to program more complex posting filter,
we're afraid.

As the last point it should be made clear, that the string you specify
as a filter is interpreted by the bourne shell for execution. It is
thus absolutely possible, to use a posting filter like that
\begin{quote}
\begin{verbatim}
PostingFilter "/bin/cat | /bin/cat ; echo ; echo testing"
\end{verbatim}
\end{quote}
even though one might argue whether this particular example is a
useful thing. Anyway, you know what we wanted to demonstrate.

\section{PGP-encrypted mailing lists}

Another very useful feature of the posting filter and the access
control languange is the ability to maintain \Def{encrypted mailing
lists}. The idea is very simple: You create a PGP key pair for your
mailing list and spread the public key among the subscribers of your
mailing list. In turn you collect their public keys and store them on
the mailing list server.

Whenever a subscriber wants to post an article to the mailing list, he
will encrypt it with the public key of the list server before
transferring it through the Internet. Petidomo will then receive the
mail, decrypt and process it and encrypt it again, with the public
keys of the subscribers. Once encrypted again, the mail is distributed
to the readers.

Please note that at no time the mail was sent through the Internet in
clear text. Hence this mode is well-suited for maintaining internal
discussion lists for, say, software development among a few people who
know each other but live spread throughout the world. Included in the
distribution are two scripts, \file{pgp-encrypt.sh} and
\file{pgp-decrypt.sh}, which realize this. The setup needs a bit of
work, but once you understand the principle, it is rather easy. Just
follow the steps described below.

\begin{enumerate}

\item Get the PGP software package from `http://www.pgpi.com/', you
will need the PGP 2.6.2 version or later --- 5.x won't work, as far as
I know, maybe someone wants to adapt the PGP-mechanism to PGP 5.x, any
volunteers are welcome, and install it.

\item Log in as user ``petidomo''.

\item Create a directory \file{.pgp} in the home directory of the
users Petidomo runs under and set the {\tt \$PGPPATH} variable to it.

\item Create a PGP key pair by calling `pgp -kg''. As user-id enter
the address of the mailing list itself, for example: ``The secret
mailing list $<$secretlist@example.org$>$''.

\item Create a \file{config.txt} file for PGP in the \file{.pgp}
directory and insert the appropriate user id there.

\item Distribute the newly created PGP key of the mailing list among
the subscribers.

\item Add the PGP keys of the subscribers to Petidomo's keyring.

\item Edit the following definitions in \file{pgp-encrypt.sh}:

\begin{quote}
\begin{verbatim}
#
# Please customize these things for your system.
#
PGP=/usr/local/bin/pgp
PASSWORD="DecryptMe"
PGPPATH=$PDHOME/.pgp
\end{verbatim}
\end{quote}

You will need to change the location of the PGP binary and insert the
password you chose for the secret key. For security reasons, the
script itself should be owned by ``petidomo'' as user- and group id,
and it should have the permission ``110'', so that only Petidomo can
execute it.

\item Edit the equivalent definitions in \file{pgp-encrypt.sh}.

\item Now create the mailing list in question. In our example that
would be ``secretlist''. Naturally the mailing list should not be open
for public subscription.

\item Edit the ACL file of the ``secretlist'' to contain the following
line:

\begin{quote}
\begin{verbatim}
if (body matches "^-----BEGIN PGP MESSAGE-----$") then
        filter "~petidomo/bin/pgp-decrypt.sh";
\end{verbatim}
\end{quote}

\item Edit the config file to have the following posting filter:

\begin{quote}
\begin{verbatim}
PostingFilter   "~petidomo/bin/pgp-encrypt.sh secretlist"
\end{verbatim}
\end{quote}

Please note that you must provide the name of the mailing list on the
command line as parameter to \file{pgp-encrypt.sh}, so that it know
which list file it should take the subscriber addresses from.

\item Do a test posting and that's it.

\end{enumerate}

There are a few things you should take care of: First of all, you must
make sure that you have the PGP public keys of all subscribers in the
keyring belonging to the ``petidomo'' user, or some of them won't be
able to decipher the mail posted via the list. You must also take care
that the addresses these people are subscribed under, are actually
listed in their public key, or PGP won't be able to recognize who is
who, when being called by \file{pgp-encrypt.sh}.

Finally, make sure that you do this only with the correct versions of
the software. Petidomo needs to be version 2.1 or later, earlier
versions won't work. The PGP binary needs to understand the {\tt -@}
operator on the command line, which has been added in PGP 2.6i at some
time.

One last hint: If PGP-encryption or decryption doesn't work, it will
usually help to remove the {\tt \$LOGFILE} parameter from the {\tt
trap} command in the scripts:

\begin{quote}
\begin{verbatim}
trap 'rm -f $TMPFILE $HEADER $BODY $NEWBODY $LOGFILE; exit'...
                                            ^^^^^^^^
\end{verbatim}
\end{quote}

As a result, the script won't delete the output PGP issued when called
after exiting. Thus you will find the file still lying in \file{/tmp}
and can easily investigate what's wrong.

\section{Virtual hosting and sendmail}
\label{virtual hosting and sendmail}

A very useful things is Petidomo's virtual hosting feature.
Unfortunately, there are a few traps into which you might run when you
are trying to use it together with the Allmann sendmail. But we'll
start at the beginning.

If you host mailing lists for domains other than your own, you can
tell Petidomo to use this name instead of the local one for certain
mailing lists in the list config file by setting the appropriate
``HostName''. Now all mail posted to this list, or sent to the
``-request'' address of the list, will appear to be coming from the
domain name you configured, rather than your own.

When you are using sendmail v8, you will have to write these names to
the \$w\$ class in your sendmail.cf file, or the corresponfing M4
config. This is done by adding the line
\begin{quote}
\begin{verbatim}
Cwdomain.name1 domain.name2 ...
\end{verbatim}
\end{quote}
to the file.

This will tell sendmail that these names are to be accepted and
delivered locally rather than to the MX of these entries.

Doing this might deliver a surprise, though, if you are using
sendmail's masquerading abilities to hide the various hostname of your
domain. Per default, sendmail masquerades not only the domain names
you told him with the MASQUERADE\_DOMAIN() command, it automatically
masquerades all domain names of the \$w\$ class, too.

The result is that Petidomo's fine virtual hosting is gone
immediately, because sendmail will re-write the name to your own
domain name. The fix for this is rather easy: Add the command
``FEATURE(limited\_masquerade)'' to your M4 file and sendmail won't
touch the names that are stated only in the \$w\$ class.

\section{Mailing list archives}
\label{mailing list archives}

If your are hosting a public mailing list, you might want to offer a
mailing list archive, that is accessible through the WWW and has all
the postings available for immediate access. We were in the midst of
developing a tool that does this for you when we came accross a
brilliant tool named ``MHonArc''. We installed it, tested it, and
deleted all attempts of writing something like that ourselves
immediately.

We strongly recommend looking at MHonArc, if you want to offer a WWW
archive of your mailing lists. You can find more information about
MHonArc at the following location:
`http://www.oac.uci.edu/indiv/ehood/mhonarc.html'

The installation of the tool itself is very easy. Once you have
MHonArc running, just enable the archiving feature in Petidomo and
feed the archives into MHonArc. That's it.

\section{Verifying the address lists}

Petidomo tries its best to make sure that only syntactically correct
addresses are subscribed to mailing lists, and if you stick to the
correct mail interface, there's very little chance, an incorrect
address will make it into the \file{list} file.

Sometimes, it is necessary to edit these files manually, though, or
Petidomo's address validation algorithm fails. Once you have an
incorrect address in your list file, sendmail will abort with an
error, without trying to deliver the mail at all.

To clarify, this does not happen when an address is not reachable,
this happens only when you subscribe something like {\tt
hey@this@is@wrong....}. Once you suspect that your address list has
been corrupted, there's an easy way to find out, which addresses are
wrong. Simply use sendmail's address verification mode like this:

\begin{quote}
\begin{verbatim}
 $ xargs <list sendmail -bv | sed -e '/deliverable/d'
 > @bogus.address.here... user address required
\end{verbatim}
\end{quote}

This call will find all incorrect address and notify you. The 'sed'
call will filter out all correct addresses for your convenience.

\end{document}
